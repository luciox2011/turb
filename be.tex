\documentclass[a4paper,12pt]{article}

\usepackage[french]{babel}
\usepackage[latin1]{inputenc}
\usepackage[T1]{fontenc}
\usepackage{amsmath}
\usepackage{amsfonts}
\usepackage{amssymb}
\usepackage[bottom]{footmisc}
\usepackage[a4paper,margin=2.5cm,footskip=.5cm]{geometry}
\usepackage{graphicx}
\usepackage{subfig}
\usepackage{morefloats}

\author{David Higuera Caubilla \and Francisco Moreira Huhn \and Garazi Gomez de Segura Solay \and Lucio Carlos Pimentel Paiva}
\title{BE de Turbulence}

\newcommand{\pt}[1]{\frac{\partial #1}{\partial t}}
\newcommand{\px}[1]{\frac{\partial #1}{\partial x}}

\begin{document}
\maketitle
\section*{Sujet 3 - Modes obliques}
\paragraph{}
Le probl�me �tudi� est l'�coulement de Poiseuille incompressible sujet � une perturbation oblique. L'�coulement est d�compos� en un champ moyen plus une fluctuation comme d�crit ci-dessous :

\begin{equation}
\begin{pmatrix}U\\V\\W\\P\end{pmatrix} = \begin{pmatrix}\overline{U}\\\overline{V}\\\overline{W}\\\overline{P}\end{pmatrix} + \begin{pmatrix}u\\v\\w\\p\end{pmatrix}
\label{eq:decomp}
\end{equation}

\end{document}
