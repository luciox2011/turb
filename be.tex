\documentclass[a4paper,11pt]{article}

\usepackage[french]{babel}
\usepackage[latin1]{inputenc}
\usepackage[T1]{fontenc}
\usepackage{amsmath}
\usepackage{amsfonts}
\usepackage{amssymb}
\usepackage[bottom]{footmisc}
\usepackage[a4paper,margin=2.5cm,footskip=.5cm]{geometry}
\usepackage{graphicx}
\usepackage{subfig}
\usepackage{morefloats}

\author{David Higuera Caubilla \and Francisco Moreira Huhn \and Garazi Gomez de Segura Solay \and Lucio Carlos Pimentel Paiva}
\title{BE de Turbulence}

\newcommand{\pt}[1]{\frac{\partial #1}{\partial t}}
\newcommand{\px}[1]{\frac{\partial #1}{\partial x}}
\newcommand{\py}[1]{\frac{\partial #1}{\partial y}}
\newcommand{\pz}[1]{\frac{\partial #1}{\partial z}}

\newcommand{\pxx}[1]{\frac{\partial^2 #1}{\partial x^2}}
\newcommand{\pyy}[1]{\frac{\partial^2 #1}{\partial y^2}}
\newcommand{\pzz}[1]{\frac{\partial^2 #1}{\partial z^2}}

\begin{document}
\maketitle
\section*{Sujet 3 - Modes obliques}
\paragraph{}
Le probl�me �tudi� est l'�coulement de Poiseuille incompressible sujet � une perturbation oblique. L'�coulement est d�compos� en un champ moyen plus une fluctuation comme d�crit ci-dessous :

\begin{equation}
\begin{pmatrix}U\\V\\W\\P\end{pmatrix} = \begin{pmatrix}\overline{U}\\\overline{V}\\\overline{W}\\\overline{P}\end{pmatrix} + \begin{pmatrix}u\\v\\w\\p\end{pmatrix}
\label{eq:decomp}
\end{equation}

Dans le cas particulier de l'�coulement �tudi�, on a :
\begin{equation}
\begin{pmatrix}U\\V\\W\\P\end{pmatrix} = \begin{pmatrix}\overline{U}(y)\\0\\0\\\overline{P}\end{pmatrix} + \mathrm{exp}\;i(\alpha x + \beta z - \omega t)\cdot\begin{pmatrix}\hat{u}(y)\\\hat{v}(y)\\\hat{w}(y)\\\hat{p}(y)\end{pmatrix}
\label{eq:modal}
\end{equation}

Les �quations de Stokes �tant v�rifi�es par l'�coulement de base, les �quations s'�crivent :

\begin{equation}
\begin{cases}
\px{u}+\py{v}+\pz{w}=0\\
\pt{u}+\overline{U}\px{u}+v\py{\overline{U}} + \frac{1}{\rho}\px{p} = \nu\left(\pxx{u}+\pyy{u}+\pzz{u}\right)\\
\pt{v}+\overline{U}\px{v} + \frac{1}{\rho}\py{p} = \nu\left(\pxx{v}+\pyy{v}+\pzz{v}\right)\\
\pt{w}+\overline{U}\px{w} + \frac{1}{\rho}\py{p} = \nu\left(\pxx{w}+\pyy{w}+\pzz{w}\right)
\end{cases}
\label{eq:perturb1}
\end{equation}

Avec la substitution de la forme modale, on obtient :

\begin{equation}
\begin{cases}
i\alpha\hat{u}+\frac{\hat{v}}{dy}+i\alpha\hat{w}=0
\end{cases}
\label{eq:perturb2}
\end{equation}
\end{document}
